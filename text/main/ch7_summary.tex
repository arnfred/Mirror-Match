\chapter{Summary}
\label{S:Summary}

This thesis has addressed the problem of matching feature points without 
using geometrical constraints, proposing \emph{Mirror Match (MM)} and 
\emph{Mirror Match with Clustering (MMC)}.  The two algorithms share the 
common idea that feature points should have better matches in another 
image than in the image they came from to be considered good matches.  
\emph{MMC} further improves on this idea by using the structure of the 
similarity graph of the feature points. 

The algorithms show promising results when tested on the \emph{Murals} 
data set. The results on 900 image pairs show that \emph{MM} and 
\emph{MMC} generally outperform existing matching algorithms 
\emph{Ratio}, \emph{Isomatch} and \emph{Spectral}. In addition \emph{MMC} 
outperforms \emph{MM}.  We show that this result generalizes to 
variations in viewpoint change by comparing \emph{MMC} to \emph{Ratio} 
over a set of image pairs with an increasing magnitude of perspective 
difference. We go on to apply the algorithms to a real life image case 
featuring occlusion and a slight viewpoint change and show that the 
performance of \emph{MM} and \emph{MMC} is consistent with the results 
on \emph{Murals}. To show that \emph{MM} or \emph{MMC}
can further improve the results of a geometrical method, we create 
\emph{Spectral-MMC} by combining \emph{Spectral} and \emph{MMC} and 
compare the three on the full \emph{Murals} dataset where it yields 
results better than both \emph{Spectral} and \emph{MMC}.

The practical application of \emph{MM} and \emph{MMC} is demonstrated by 
applying the algorithms to the problems of near duplicate detection and 
for the case of \emph{MMC} to panorama recognition and alignment. For 
the case of near duplicate detection, \emph{MMC} provides an 8.02\% 
improvement over \emph{Ratio} in a near duplicate detection framework 
making use of local features. Similarly for panorama recognition and 
alignment a simple panorama stitching method based on \emph{MMC} 
succeeds in cases where \emph{Autostitch} is unable to recognize and 
align the source images correctly. Both applications illustrate the 
central role that performance of the matching algorithm has on tasks 
that makes use of local features to match image pairs with potential 
overlap.
%
